
% Honours Report Template
% updated May 2013
%
\documentclass[a4paper,12pt]{article}
%\makeatletter
%\renewcommand\paragraph{\@startsection{paragraph}{4}{\z@}%
%{-2.5ex\@plus -1ex \@minus -.25ex}%
%{1.25ex \@plus .25ex}%
%{\normalfont\normalsize\bfseries}}
%\makeatother
%\setcounter{secnumdepth}{4} % how many sectioning levels to assign numbers to
%\setcounter{tocdepth}{4}    % how many sectioning levels to show in ToC
%
\usepackage[section]{placeins}
\usepackage{pbox}
\usepackage{epsfig}
\usepackage{latexsym}
\usepackage{graphicx}
\usepackage[square,numbers]{./natbib/natbib}
\usepackage{url}
\usepackage{caption}
\usepackage{subcaption}
% It also sets the bibliographystyle to plainnat; for more information on
% natbib citation styles, see the natbib documentation, a copy of which
% is archived at http://www.jmlr.org/format/natbib.pdf
\usepackage{setspace}
\usepackage{amsmath}
\usepackage{amssymb}
\usepackage{amsfonts}
\usepackage{color}

%\graphicspath{{./figures/}
%Formatting-------------------------------------------------------------------
%\renewcommand{\refname}{\textbf{Literature}}
%
\renewcommand{\contentsname}{\small\textbf{{\center Table of Contents}}}
%
\setlength{\textheight}{8.8in}
%
\setlength{\topmargin}{-1.5cm}
%
\doublespacing
%\setlength{\textwidth}{17cm}
%
%\setlength{\oddsidemargin}{-0.1714in}
%
% Boxit -----------------------------------------------------------
\setlength{\fboxrule}{0.2mm} \setlength{\fboxsep}{4mm}
%
\newsavebox{\savepar}
\newenvironment{boxit}{\begin{lrbox}{\savepar}
        \begin{minipage}[b]{4.6in}}
        {\end{minipage}\end{lrbox}\fbox{\usebox{\savepar}}}
        
        
 \hyphenation{op-tical net-works Mathe-ma-tical street-scape street-scapes aes-the-tics aes-the-tic com-pu-ting geo-metric Geo-me-tric geo-metry boun-da-ries de-ve-lop-ment know-ledge mani-fold mani-folds high-di-men-sio-nal}
%
%
%
% Document-----------------------------------------------------------------------
%
\begin{document}
%
\title{\bf Technology \& User Engagement}
%
\author{Ross Bille\\
School of Electrical Engineering \& Computer Science\\
The University of Newcastle\\ Callaghan NSW 2308, Australia\\
Email: \texttt{c3127333@uon.edu.au} } 
\maketitle


\newpage
\begin{abstract}%
\noindent Document abstract 
\end{abstract}

\pagebreak

\tableofcontents

\pagebreak

\listoffigures        

\pagebreak

\section{Introduction}
%User engagement and why we need it
Systems are created daily to achieve a set of goals, these goals vary greatly depending on the environment that the system has to run in, be it in the workplace, schools, community, or general commercial systems. Regardless of the environment, the types of systems that will be discussed in this paper all share one thing, that is to achieve their goals they all require users to interact with the system in one way or another. 
This interaction needs to be present throughout the entire life cycle of the system; however this paper will mainly discuss two phases of this life cycle, the initial implementation and continued execution of the system. 
This paper will discuss what should be present during the initial implementation of a system, in order to get users interested in interacting with the system, and what techniques we can use to keep users motivated through the continued execution of a system. In 2006~\citet[p.~1]{fun-of-use} mentions ``Successful software products should arouse positive emotions'', thus further outlining another aim of software systems in particular that when met can assist in motivation~\citep{fun-of-use}.

\subsection{The City Evolutions Project}
%introduction to city evolutions project
Newcastle City Council (NCC) were interested in increasing tourism through Newcastle and in particular through Watt Street. To do this NCC in collaboration with Newcastle Now and the University of Newcastle (UON) developed the City Evolutions Project. The main idea of this project was to promote Newcastle's rich cultural and historical aspects. 
The project required ten sites to be set up through Watt Street each containing a projector with some sites also housing a computer, a wireless hotspot, and a motion detecting camera. The aim being that material expressing the cultural and historical aspects of Newcastle was projected onto the sides of buildings between sunset and 10pm every day.\\
In order for this project to be a success the displayed content needs to interest and attract people to Watt street and also motivate them to keep coming back, thus ultimately increasing tourism. Which brings us into the reason for the research that this paper presents.

\section{Current Fields Utilising Technology for User Engagement}\label{sec:current-fields}
The following sections introduce some areas of everyday life that have implemented systems which rely on user engagement for their success. We also discuss these systems and what goals they set out to achieve.

\subsection{Workplace}

In the workplace systems have been introduced in order to share information (such as health plan offerings~\citep{taxonomy-of-gamification})  throughout the company, and increase worker productivity measured through key performance indicators (KPI's).
Some systems include (but not limited to):
\begin{itemize}
	\item{Bonus Schemes}
	\item{Online portals~\citep{taxonomy-of-gamification}}
	\item{Inter-office games~\citep{taskville}}
\end{itemize}

\subsubsection{Taskville}\label{sec:taskville}
Taskville is one such system designed to be implemented in the workplace~\citep{taskville}. Taskville was designed to ``address key challenges in contemporary distributed and diverse workplaces''~\citep[p.~4]{taskville}. 
It does this by implementing a game that is to be played by the different groups of an organisation, where each group  has a city in the game that they must grow and control. Taskville implements gamification (see section~\ref{sec:gamification}) techniques such as rewards (see section~\ref{sec:rewards}) in form of in-game currency that can be spent on buildings, and points that are used to define who is the mayor of each city, both of which can be earned by meeting KPI's or completing defined tasks.\\ Taskville was found to increase the awareness of the work performed by other groups within the workplace and at the same time encouraged each individual to outperform other group members in order to reach the title of mayor~\citep{taskville}.

\subsubsection{Agentville}\label{sec:agentville}
Agentville is a tool developed to assist employees of a call center in managing their KPI's~\citep{production-environments}. As centers already monitor statistics such as:
\begin{itemize}
	\item{Average Handle Time (AHT)~\citep{production-environments},}
	\item{Occupancy~\citep{call-center},}
	\item{Abandoned calls~\citep{call-center},}
	\item{Calls answered~\citep{call-center},}
	\item{Calls offered~\citep{call-center},}
	\item{Average wait time~\citep{call-center}, and}
	\item{Grade of service~\citep{call-center}}
\end{itemize}
the application was not developed to create more statistics to monitor but to help each individual keep track of their own statistics in relation to the rest of the organisation~\citep{production-environments}.

\subsection{Education}\label{sec:education}
Educational institutes implements many different systems with many different aims, two such aims are: handling behavior and motivating students to learn.

\par
Advancements in technology have led to it being adopted within the educational sector~\citep{distance-education}, this has led to technologies such as the Internet and virtual reality (VR) being adopted in all forms of education~\citep{virtual-reality}.
The following sections discuss what systems have been implemented throughout a few aspects of education, from simple paper-based systems to advances VR systems.

\subsubsection{Classroom}\label{sec:classroom}
Walk into almost any primary school classroom and you will see a poster with a list of students names with stickers next to these names. This is a system that teachers use to reward students and publicly display achievements to the class, this is done to encourage competition between the students, which in turn motivates them to do better~\citep{school-kids}.

\subsubsection{Online Learning}
With the advancements in technology over the past few years universities around the world have started offering courses by distance in an online manner; in addition to this, there are many websites dedicated to teaching content to anyone that wants to learn it.

\paragraph{\indent~University Distance Education.}
Generally with higher education there is less emphasis on the initial interest of the student and more on keeping the student motivated once they have enrolled in a course, this is due to the content itself interesting the students. Unfortunately keeping students motivated after their initial interest wears off is the difficult part, this becomes more apparent when it comes to distance education~\citep{distance-education}.
Generally the students that enroll in courses online are of mature age and sometimes they have other focuses in life~\citep{distance-education}.

\par
Universities around the world have been taking steps to increase this motivation, largely through the use of multimedia, lectures can now be pre-recorded and streamed to students all around the world. Universities around the world, have adopted this technique, however this is still not enough and this paper will describe some more techniques that would be of great benefits to this type of education (see section~\ref{sec:techniques-to-enhance}).

\paragraph{\indent~Educational Websites.} 
Coursera is an online collaboration of universities such as (but not limited to): 
\begin{itemize}
	\item{Stanford University,}
	\item{Yale University,}
	\item{The University of Tokyo, and}
	\item{HEC Paris}
\end{itemize}
Coursera aims to create a collaborative environment for users to learn and teach each other. The universities that contribute to Coursera use the above mentioned technique of streaming lectures to students and users; however there are a few more techniques that are implemented in order to assist in motivating students, one of these techniques is the use of social media to share achievements which will be explained in more depth in sections~\ref{sec:gamification} and~\ref{sec:social-media}.

\par
Another educational website, Codecademy specialises

\subsection{Therapy}
Used to help patients open up and express themselves
\subsection{Commercial}
Used to create and maintain a large customer base
Starbucks coffee rewards etc
\subsection{Community}
Increase tourism and community awareness (general well-being)

\section{Techniques to Enhance User Engagement}\label{sec:techniques-to-enhance}

\subsection{Rewards}\label{sec:rewards}

\subsubsection{Extrinsic}
\subsubsection{Intrinsic}

\subsection{The Skinner Box}
The Skinner Box is a technique named by B.F. Skinner describing a small cage which houses a small animal, a lever and a way of delivering rewards to the animal~\citep{openingSkinersBox}. The idea was to train the animal to continuously hit the lever. The animal gets rewarded at random intervals for lever pressing, with the possibility of punishment if the lever is not pressed (usually by electric shock). This technique has been around since the 1930’s and has been implemented in video games since their inception~\citep{theSkinnerBox}.
\\
Game designers have expanded the Skinner Box concept and made it feel like second nature to players. They have done this by offering virtual rewards to players who achieve pre-defined goals (this is obvious in any games that have levels and/or virtual currency) and recently by punishing players for not playing regularly (usually by devaluing the players hard earned virtual currency or by making in-game possessions deteriorate).
\\
The Skinner Box concept is a great way to develop players’ obsession to a game while offering no real world reward, often leaving the player wondering why they wasted so much time on the game~\citep{fiveCreepyWays} 

\subsection{Gamification}\label{sec:gamification}
Gamification is the process of integrating game dynamics and techniques into non-game processes. This is usually done to increase interaction, productivity or satisfaction.
\subsubsection{The History of Gamification}
\subsubsection{Elements in Game Design}
Gamification takes advantage of various features that are present in game design. The following paragraphs discuss the effectiveness of these features as well as the importance of their potential applications in enhancing user interactions beyond the gaming culture. 
\paragraph{\indent~Rewards}
\paragraph{\indent~Competition}
\paragraph{\indent~Stories}

\subsection{Virtual Reality}
\subsection{Effort Reduction}
\subsection{Usability}
\subsection{Social Media}\label{sec:social-media}
\subsection{Education}
%closing the loop etc
%o
\newpage
\section{City Evolutions}
\subsection{Analysis}
Analyse the current setup of the city evolutions project.\\
discuss the current statistics:
\begin{itemize}
	\item{Good points (project clearly had a boost on open)}
	\item{Bad points (users aren't motivated to keep coming back and scanning the qr code because there isnt a lot of new material)}
\end{itemize}

\begin{figure}[ht!]
	\centering
	\includegraphics[width=150mm]{./images/OverallQRCodeDownload}
	\caption{QR Code - Overall scans}
	\label{QR-overall-access-overTime}
\end{figure}

\begin{figure}[ht!]
	\centering
	\includegraphics[width=50mm]{./images/OSAccess}
	\caption{QR Code - OS spread}
	\label{QR-OS-access}
\end{figure}

\begin{figure}[ht!]
	\centering
	\includegraphics[width=150mm]{./images/threkeld-scans}
	\caption{QR Code - Threlkeld Speak scans}
	\label{QR-threkeld}
\end{figure}

\begin{figure}[ht!]
	\centering
	\includegraphics[width=50mm]{./images/qrcode-threlkeld}
	\caption{QR Code - Threlkeld Speak}
	\label{QR-threkeld-barcode}
\end{figure}

\begin{table}
	\centering
	\begin{tabular}{|c|c|c|}\hline
		\textbf{Application} & 	\textbf{Number of Scans} & \textbf{Interactivity (see table~\ref{table:interactivity-description})} \\\hline
		Macquaries Chest & 2 & Low\\
		Car Culture & 3 & None\\
		Lansescape & 34 & Medium\\
		Coal Collector & 36 & Medium\\
		Dreaming & 3 & None\\
		Settling In & 28 & Low\\
		Major Morisset & 36 & Low\\
		Evolutions & 6 & None\\
		Shanghai & 2 & Medium\\
		Boom \& Bust & 2 & Low\\
		Threlkeld 	&	132 & Medium\\\hline
	\end{tabular}
	\caption{City Evolutions Applications and their popularity}
	\label{table:application-popularity}
\end{table}

\begin{table}
	\centering
	\bgroup
	\def\arraystretch{2.5}%  1 is the default, change whatever you need
	\begin{tabular}{|l|l|}\hline
		\textbf{Level} 	& \textbf{Description}\\\hline
		None 	& \pbox{20cm}{No user input is taken at all, generally a slideshow or movie} \\\hline
		Low 	& \pbox{20cm}{Generally a single has basic control over the application\\ 
				  			  i.e. User makes movement to trigger the application to take action.} \\\hline
		Medium 	& \pbox{20cm}{Users have basic control over the application and are given constant\\
							  feedback, the more users the more feedback is received.}\\\hline
		High 	& \pbox{20cm}{Users have fine grained control over the application,\\ 
							  with constant feedback and a reason to come back to the application\\
							  i.e. scores are saved, competitions are run, etc.}\\\hline
	\end{tabular}
	\egroup
	\caption{Description of interactivity levels used in \ref{table:application-popularity}}
	\label{table:interactivity-description}
\end{table}

\newpage
\subsection{Proposal}

\begin{figure}
\centering
\begin{minipage}{.5\textwidth}
  \centering
  \includegraphics[width=.4\linewidth]{./images/iphone-interface}
  \captionof{figure}{Application start screen}
  \label{iphone-interface-startscreen}
\end{minipage}%
\begin{minipage}{.5\textwidth}
  \centering
  \includegraphics[width=.4\linewidth]{./images/iphone-interface-menu}
  \captionof{figure}{Application Menu}
  \label{iphone-interface-menu}
\end{minipage}
\end{figure}


\begin{figure}
\centering
\begin{minipage}{.5\textwidth}
  \centering
  \includegraphics[width=.4\linewidth]{./images/iphone-interface-leaderboard}
  \captionof{figure}{Application leader board}
  \label{iphone-interface-leaderboard}
\end{minipage}%
\begin{minipage}{.5\textwidth}
  \centering
  \includegraphics[width=.4\linewidth]{./images/iphone-interface-map}
  \captionof{figure}{Application locations map}
  \label{iphone-interface-map}
\end{minipage}
\end{figure}

\begin{figure}[ht!]
	\centering
	\includegraphics[width=100mm]{./images/MultiPlayerChurchGame}
	\caption{Multiplayer pong-style game to be played at the church.}
	\label{application-multiPlyerPong}
\end{figure}

\newpage
\section{Discussion}
%
Give an extended and detailed discussion of your study. Explain what we can learn from your study.
%

\section{Conclusion}
%
A brief final summary of the main achievements and outcomes. Possibly some suggestions for future work that can follow on from your project.%

\section{Glossary}
%
\subsection*{Acknowledgements}
The author is grateful to ....
%
\vskip 0.2in
\newpage
\bibliographystyle{apalike}
\bibliography{./literature.bib}

\end{document}
