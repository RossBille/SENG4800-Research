
% Honours Report Template
% updated May 2013
%
\documentclass[a4paper,12pt]{article}
%\makeatletter
%\renewcommand\paragraph{\@startsection{paragraph}{4}{\z@}%
%{-2.5ex\@plus -1ex \@minus -.25ex}%
%{1.25ex \@plus .25ex}%
%{\normalfont\normalsize\bfseries}}
%\makeatother
%\setcounter{secnumdepth}{4} % how many sectioning levels to assign numbers to
%\setcounter{tocdepth}{4}    % how many sectioning levels to show in ToC
%
\usepackage[section]{placeins}
\usepackage{pbox}
\usepackage{epsfig}
\usepackage{latexsym}
\usepackage{graphicx}
\usepackage[square,numbers,sort&compress]{./natbib/natbib}
\usepackage{url}
\usepackage{caption}
\usepackage{subcaption}
% It also sets the bibliographystyle to plainnat; for more information on
% natbib citation styles, see the natbib documentation, a copy of which
% is archived at http://www.jmlr.org/format/natbib.pdf
\usepackage{setspace}
\usepackage{amsmath}
\usepackage{amssymb}
\usepackage{amsfonts}
\usepackage{color}

%\graphicspath{{./figures/}
%Formatting-------------------------------------------------------------------
%\renewcommand{\refname}{\textbf{Literature}}
%
\renewcommand{\contentsname}{\small\textbf{{\center Table of Contents}}}
%
\setlength{\textheight}{8.8in}
%
\setlength{\topmargin}{-1.5cm}
%
\doublespacing
%\setlength{\textwidth}{17cm}
%
%\setlength{\oddsidemargin}{-0.1714in}
%
% Boxit -----------------------------------------------------------
\setlength{\fboxrule}{0.2mm} \setlength{\fboxsep}{4mm}
%
\newsavebox{\savepar}
\newenvironment{boxit}{\begin{lrbox}{\savepar}
        \begin{minipage}[b]{4.6in}}
        {\end{minipage}\end{lrbox}\fbox{\usebox{\savepar}}}
        
        
 \hyphenation{op-tical net-works Mathe-ma-tical street-scape street-scapes aes-the-tics aes-the-tic com-pu-ting geo-metric Geo-me-tric geo-metry boun-da-ries de-ve-lop-ment know-ledge mani-fold mani-folds high-di-men-sio-nal}
%
%
%
% Document-----------------------------------------------------------------------
%
\begin{document}
%
\title{\bf Technology \& User Engagement}
%
\author{Ross Bille\\
School of Electrical Engineering \& Computer Science\\
The University of Newcastle\\ Callaghan NSW 2308, Australia\\
Email: \texttt{c3127333@uon.edu.au} } 
\maketitle


\newpage
\begin{abstract}%
\noindent User engagement is necessary for any system that requires any form of input from a user. 
This raises many questions such as `how can a system interest a user', `how can a system motivate a user', and `how can a system educate a user'. 
Whats more difficult is creating a system that can answer all three of these questions, this is due to the relationship between education and motivation.
This paper will discuss methods and techniques that can be used to aid in achieving these goals.
\end{abstract}

\pagebreak

\tableofcontents

\pagebreak

\listoffigures        
\listoftables

\pagebreak

\section{Introduction}
%User engagement and why we need it
Systems are created daily to achieve a set of goals, these goals vary greatly depending on the environment that the system has to run in, be it in the workplace, schools, community, or general commercial systems.
Regardless of the environment, the types of systems that will be discussed in this paper all share one thing, that is to achieve their goals they all require users to interact with the system in one way or another. 
This interaction needs to be present throughout the entire life cycle of the system; however this paper will mainly discuss two phases of this life cycle, the initial implementation and continued execution of the system. 
This paper will discuss what should be present during the initial implementation of a system, in order to get users interested in interacting with the system, and what techniques we can use to keep users motivated through the continued execution of a system. 
In 2006~\citet[p.~1]{fun-of-use} mentions ``Successful software products should arouse positive emotions'', thus further outlining another aim of software systems in particular that when met can assist in motivation~\citep{fun-of-use}.

\subsection{The City Evolutions Project}\label{sec:city-evolutions-project}
%introduction to city evolutions project
Newcastle City Council (NCC) were interested in increasing tourism through Newcastle and in particular through Watt Street. To do this NCC in collaboration with Newcastle Now and the University of Newcastle (UON) developed the City Evolutions Project. 
The main idea of this project was to promote Newcastle's rich cultural and historical aspects. 
The project required ten sites to be set up through Watt Street each containing a projector with some sites also housing a computer, a wireless hotspot, and a motion detecting camera. The aim being that material expressing the cultural and historical aspects of Newcastle was projected onto the sides of buildings between sunset and 10pm every day.\\
In order for this project to be a success the displayed content needs to interest and attract people to Watt street and also motivate them to keep coming back, thus ultimately increasing tourism. Which brings us into the reason for the research that this paper presents.


\section{Techniques to Enhance User Engagement}\label{sec:techniques-to-enhance}
The following section will discuss some methods and techniques that can be used in order to increase user engagement and their satisfaction with the system.

\subsection{Rewards}\label{sec:rewards}
%have to change after swapping sections around
Rewards have been utilised in many areas of every days life (see section~\ref{sec:current-fields}) as a means for motivation throughout all types of systems currently in operation today.
This section aims to divide the set of rewards into two subsets, extrinsic and intrinsic~\citep{bread-and-games,fun-of-use,deci_extrinsic_2001,McGonigal:2011:RBW:1972527}, in order to obtain a better understanding of how rewards should be utilised in the systems implemented in future.

\subsubsection{Extrinsic}\label{sec:extrinsic}
Extrinsic rewards are those which are external to the user~\citep{deci_extrinsic_2001,McGonigal:2011:RBW:1972527} and can be rewards such as: monetary,	in-game currency, score, entries in a leader board, promotions, leveling up, and trophies~\citep{bread-and-games}.

\subsubsection{Intrinsic}\label{sec:intrinsic}
Intrinsic rewards are more personal to the user~\citep{patient-education-and-training}, some examples of intrinsic rewards are as follows: enjoyment, Challenge, Excitement, and Education~\citep{bread-and-games}.

\subsubsection{Summary}
To summarise, rewards are an essential part of motivating a user to engage with a system; however care needs to be taken with what types of rewards are selected. It is important not to implement an unreasonable amount of extrinsic rewards as this can leave a user feeling bought and unaccomplished~\citep{openingSkinersBox,theSkinnerBox,McGonigal:2011:RBW:1972527,bread-and-games} sometimes making playing a game feel like doing work~\citep{turning-play-into-work}. 
Instead extrinsic rewards should be added to complement intrinsic rewards and thus making them feel intrinsic as well. 
This all helps to create a system that increases satisfaction in the user, motivates a user to achieve higher goals within the system, provides higher productivity, and keeps the user feeling free and competent~\citep{intrinsic-motivation}.

\subsection{The Skinner Box}
The Skinner Box is a technique named by B.F. Skinner describing a small cage which houses a small animal, a lever and a way of delivering rewards to the animal~\citep{openingSkinersBox}. 
The idea was to train the animal to continuously hit the lever. 
The animal gets rewarded at random intervals for lever pressing, with the possibility of punishment if the lever is not pressed (usually by electric shock). 
This technique has been around since the 1930’s and has been implemented in video games since their inception~\citep{theSkinnerBox}.

\par
Game designers have expanded the Skinner Box concept and made it feel like second nature to players. 
They have done this by offering virtual rewards to players who achieve pre-defined goals (this is obvious in any games that have levels and/or virtual currency) and recently by punishing players for not playing regularly (usually by devaluing the players hard earned virtual currency or by making in-game possessions deteriorate).

\par
The Skinner Box concept is a great way to develop players’ obsession with a game while offering no real world reward, instead just extrinsic rewards, often leaving the player wondering why they wasted so much time on the game~\citep{fiveCreepyWays,bread-and-games}. 

\subsection{Gamification}\label{sec:gamification}
Gamification is the process of integrating game dynamics and techniques into non-game processes. This is usually done to increase users' interaction, productivity or satisfaction.

\subsubsection{The History of Gamification}

\subsubsection{Elements in Game Design}
Gamification takes advantage of various features that are present in game design. The following paragraphs discuss the effectiveness of these features as well as the importance of their potential applications in enhancing user interactions beyond the gaming culture. 

\par
\textbf{Rewards}\\
Rewards have already been discussed thoroughly in section~\ref{sec:rewards}, here we discuss how rewards are present in the gaming environment and how they can be used in other systems. 
Within a gaming context a player is rewarded throughout the entire game for an array of tasks that differ between the types of game such as role playing games (RPGs), racing games, first person shooters {FPS}, and many more which are summarised in table~\ref{table:game-rewards}.

\begin{table}[!ht]
	%\centering
	\begin{tabular}{|l|l|l|}\hline
		\textbf{Game type} & 	\textbf{Activity} & \textbf{Reward} \\\hline
		Racing 	&	Defeating other racers 	& In-game currency, leader board\\
		Racing	&	Beating set lap times	& Achievements, badges\\
		RPGs 	&  	Completing quests & In-game currency/items\\
		RPGs 	&	Defeating other players & Leader board\\
		RPGs 	& 	Completing tasks X times & Experience points, in-game currency\\
		FPS 	&	Killing an enemy	& Experience points, in-game currency\\
		FPS 	&	Killing X enemies	& Achievements, in-game currency\\
		Online 	&	Logging in X days in a row & Achievements, leader board\\\hline
	\end{tabular}
	\caption{Rewards for activities commonly found in games.}
	\label{table:game-rewards}
\end{table}

Rewards within other environments need to be considered on a case-by-case basis because generic rewards tend to be extrinsic~\citep{deci_extrinsic_2001}. Rewards should be thought of as an addition to assist in motivating a user, not the sole motivator.

\par
\textbf{Competition}\\
Humans are naturally competitive beings and games leverage this in order to keep us motivated~\citep{bread-and-games}. 
Competition usually leads to one of two things, the intrinsic reward or extra motivation to try harder next time.

\par
\textbf{Stories}\\
Stories can engage people both emotionally and intellectually, while feeding off our natural desire to know how the story ends.
They can engage the user through fantasy and suspense which is why stories are a key motivational aspect of games~\citep{Narrative-Centered-Learning-Environments}.\\
These same stories can assist in motivating users to engage with any type of application or system~\citep{applications-as-stories}. 
They can help to take alleviate the pressure on a user from an system that is overwhelmingly rich in content by breaking it up and guiding the user through the system~\citep{applications-as-stories,brown_human-computer_1998,lidwell_universal_2010}.

\par
\textbf{Avatars}\\
Avatars are in-game representations of a user. 
Research by~\citet{self-modeling} has shown that avatars that resemble the user can be highly effective at motivating that user to behave in a desired way, thus another key technique in enhancing user engagement~\citep{bread-and-games}.

\subsection{Virtual Reality}\label{sec:virtual-reality}
\citet{vr-unity3d} defines virtual reality (VR) as ``a technology which allows a user to interact with a computer-simulated environment''. 
This simulated environment can be used as an educational tool in the same way that field trips are used in learning, it allows the user to directly experience the material that they are learning~\citep{virtual-reality}.
This concept can also be used as a tool to enhance user engagement because it provides a natural interface~\citep{virtual-reality}, for example users do not need to think about which button needs to be pressed to to perform an action. 
This in turn leads to a reduction in the effort required by users to interact with the system and an increase in satisfaction with the system~\citep{virtual-reality,vr-unity3d}.

\subsection{Effort Reduction}\label{sec:effort-reduction}
Effort reduction is concerned with making making interaction with a system as simple as possible. 
Effort reduction can be achieved through a variety of techniques, both software and hardware related.

\par 
One of the easiest methods to implement a system that is easy to use is the utilisation of familiar input devices, this cuts out the need for a user to re-train in order to interact with the system. 
An example of a familiar input device would be a smartphone, as~\citet{smartphone-use-in-au} has shown that 37\% of Australians have these at their disposal. 
Another example would be the use of input devices that track user movement with  technologies such as gps, motion detection, and voice recognition. 
The use of these input devices takes the strain off the user because they solely need to focus on tasks they carry out in their day-to-day lives.

\subsubsection{Usability}
Usability refers to the software interface itself and how intuitive the interface is designed.
Research into educational software in particular shows that proper interface design can aid in achieving the aims of the software. This is because the user is less concerned on the use of the system and more concerned with the content it provides~\citep{472682,5716746,enhancing-learnability}.

\subsection{Social Media}\label{sec:social-media}
Use of social networking  websites can aid in motivating users to interact with a system and has been used in conjunction with other techniques, namely gamification.
Achievements and leader boards can be published to websites such as facebook and twitter. 
This can be used as a form of advertisement, which can assist in raising awareness of the system to non-current users. Users can `share' their achievements wit their friends, who can then in turn `like' and `comment' on these achievements~\citep{6567325}. 
Users can also invite their friends to start using the application.
Making these achievements and leader boards public can also assist in motivating current users to continue using the system, due to the competitive nature brought up in section~\ref{sec:gamification}.

\subsection{Education}
A system needs to have an aim for it to be motivational~\citep{bread-and-games,deci_extrinsic_2001,addition-to-creation}, one of the common aims found throughout systems is the aim to educate. When a system can educate a user, that user is intrinsically rewarded and satisfied with the system~\citep{bread-and-games}.
Educating a user is not always easy and this section aims to clarify a technique to assist in achieving this aim.

\subsubsection{Closing the Loop}
Closing the loop refers to teaching a skill through attempting actions and receiving feedback before attempting the action again before the skill is mastered~\citep{applications-as-stories,virtual-reality,Skill-atoms}. 
\citet{Skill-atoms} defines skill atoms that assist in achieving this loop of interaction. 
These skill atoms are smaller subsets of the central skill that is being taught ultimately feeding into it.

\begin{figure}[ht!]
	\centering
	\includegraphics[width=100mm]{./images/skill-atom}
	\caption{A Skill Atom~\citep{Skill-atoms}}
	\label{skill-atom}
\end{figure}

\par
Using this closed-loop model of interaction it can ensure that a user is not overwhelmed by an application that is trying to teach a skill.
Instead users are taught small skills that are combined into sets of skills that feed into the main or centralised skill that the system is teaching~\citep{applications-as-stories,Skill-atoms}.

\subsection{Summary}
In order for a system to be motivating; it should offer some kind of intrinsic reward to the user, which is easier to implement if the system already has a meaningful purpose that would interest the user within itself. 
For example, systems that aim to educate, assists a worker in their job or assists a user to do their day-to-day tasks.

\par
A system should be interesting, in order to do this new content should be added regularly. 
The system needs an end design, that is; a system shouldn't be designed with the main aim of encouraging users to use it, because if the user doesn't get anything out of it then there is not reason for the user to interact with the system. 
In addition to the point of the system, it is important to have enough interesting content so the user is not constantly thinking about what they are trying to achieve. 
There is a delicate balance here between being to the point and being distracting.

\par
The system must not be difficult to use, users should not need to learn a new technology or interface to interact with the system. 
Murton has shown that 37\% of Australians have a smartphone, with countries such as Singapore showing a penetration of 62\%~\citep{smartphone-use-in-au}, this provides a basic tool to be utilising in these systems.

\par
Humans have a natural desire to compete~\citep{bread-and-games}, this can be leveraged in the systems we design through tools such as leader boards and achievements, in order to make a system naturally motivating.
Social media is a great way to advertise these achievements to bring the competition to a grander scale; however care should be taken to avoid slanderous comments from non-constructive individuals, usually taken care of by giving the user control over who can see their updates.

\section{Current Fields Utilising Technology for User Engagement}\label{sec:current-fields}
The following sections introduce some areas of everyday life that have implemented systems which rely on user engagement for their success. 
I will also discuss these systems and what goals they set out to achieve and how they have achieved them.

\subsection{Workplace}

In the workplace systems have been introduced in order to share information (such as health plan offerings~\citep{taxonomy-of-gamification})  throughout the company, and increase worker productivity measured through key performance indicators (KPI's).
Some systems include (but not limited to):
\begin{itemize}
	\item{Bonus Schemes}
	\item{Online portals~\citep{taxonomy-of-gamification}}
	\item{Inter-office games~\citep{taskville}}
\end{itemize}

\subsubsection{Taskville}\label{sec:taskville}
Taskville is one such system designed to be implemented in the workplace~\citep{taskville}. Taskville was designed to ``address key challenges in contemporary distributed and diverse workplaces''~\citep[p.~4]{taskville}. 
It does this by implementing a game that is to be played by the different groups of an organisation, where each group  has a city in the game that they must grow and control. 
Taskville implements gamification (see section~\ref{sec:gamification}) techniques such as rewards in the form of in-game currency and points, both of which can be earned by meeting KPI's or completing defined tasks.\\ 
Taskville was found to increase the awareness of the work performed by other groups within the workplace and at the same time encouraged each individual to outperform other group members in order to reach the title of mayor~\citep{taskville}.

\subsubsection{Agentville}\label{sec:agentville}
Agentville is a tool developed to assist employees of a call center in managing their KPI's~\citep{production-environments}. 
As call centers already monitor statistics such as: Average Handle Time (AHT)~\citep{production-environments},	occupancy, abandoned calls,	calls answered,	calls offered, average wait time, and grade of service~\citep{call-center}. 
The application was not developed to create more statistics to monitor but to help each individual keep track of their own statistics in relation to the rest of the organisation~\citep{production-environments}.

\subsection{Education}\label{sec:education}
Educational institutes implement many different systems with many different aims, two such aims are: handling behavior and motivating students to learn~\citep{deci_extrinsic_2001}.

\par
Advancements in technology have led to it being adopted within the educational sector~\citep{distance-education}, this has led to technologies such as the Internet and virtual reality (VR) being adopted in all forms of education~\citep{virtual-reality}.
The following sections discuss what systems have been implemented throughout some aspects of education, from simple paper-based systems to advances VR systems.

\subsubsection{Classroom}\label{sec:classroom}
Walk into almost any primary school classroom and you will see a poster with a list of students names with stickers next to these names. 
This is a system that teachers use to reward students and publicly display achievements to the class. 
This is done to encourage competition between the students, which in turn motivates them to do better~\citep{school-kids}.

\subsubsection{Online Learning}
With the advancements in technology over the past few years universities around the world have started offering courses by distance in an online manner; in addition to this, there are many websites dedicated to teaching content to anyone that wants to learn it.

\par
\textbf{University Distance Education}\\
Generally within higher education there is less emphasis on the initial interest of the student and more on keeping the student motivated once they have enrolled in a course, this is due to the content itself being responsible for initially interesting the students. Unfortunately keeping students motivated after their initial interest wears off is the difficult part, this becomes more apparent when it comes to distance education~\citep{distance-education}.
Generally the students that enroll in courses online are of mature age and sometimes they have other focuses in life~\citep{distance-education}.

\par
Universities around the world have been taking steps to increase this motivation, largely through the use of multimedia; lectures can now be pre-recorded and streamed to students all around the world. 
Universities around the world have adopted this technique, however retaining student interest and motivation via distance learning remains an ongoing challenge. 
This paper will further describe some techniques that would be of great benefit to this type of education (see section~\ref{sec:techniques-to-enhance}).

\par
\textbf{Educational Websites}\\ 
Coursera is an online collaboration of universities including: 
\begin{itemize}
	\item{Stanford University;}
	\item{Yale University;}
	\item{The University of Tokyo; and}
	\item{HEC Paris}
\end{itemize}
Coursera aims to create a collaborative environment for users to learn and teach each other. The universities that contribute to Coursera use the above mentioned technique of streaming lectures to students and users. Additionally there are more techniques that are implemented in order to assist in motivating students, one of these techniques is the use of social media to share achievements which will be explained in more depth in sections~\ref{sec:gamification} and~\ref{sec:social-media}.

\par
Another educational website, Codecademy, specialises in teaching programming skills in various languages. A student is given the option to choose what they want to learn and they can set their own pace, giving the students a sense of autonomy which has been found to assist with intrinsic motivation~\citep{bread-and-games}. 
Codecademy also implements a few gamification techniques (see section~\ref{sec:gamification}) such as badges (Figure~\ref{codecademy-badges}) mixed with social media aspects (see section~\ref{sec:social-media}) to keep the student coming back day after day.	

\begin{figure}[!ht]
	\centering
	\begin{minipage}{.5\textwidth}
	  \centering
	  \includegraphics[width=.6\linewidth]{./images/codecademy-badge-500exercises}
	  \label{codecademy-badge-500exercises}
	\end{minipage}%
	\begin{minipage}{.5\textwidth}
	  \centering
	  \includegraphics[width=.6\linewidth]{./images/codecademy-badge-maxstreak}
	  \label{codecademy-badge-maxstreak}
	\end{minipage}
	\caption{Codecademy badges}
	\label{codecademy-badges}
\end{figure}

\subsection{Therapy}
Researchers have recently been looking into alternate methods for delivering cognitive behavioral therapy to patients, with recent studies showing that computerised methods compare very well with the traditional face to face methods~\citep{games-for-behavior-change}. 
One of the reasons identified in~\citep{games-for-behavior-change} is the ``difficulty or tediousness of using the tools''[p.~2], which is one of the issues this paper aims to address (specifically see section~\ref{sec:effort-reduction}).

\subsubsection{Beating the Blues}
Beating the Blues is a multimedia cognitive behavioral therapy application~\citep{beating-the-blues}.
The application was designed to provide self-help to therapy patients while allowing the allocating therapist to track the patients progress.
The application was found to have a high acceptance rate due to the high levels of interaction present in the application~\citep{beating-the-blues,games-for-behavior-change}. 


\subsection{Commercial}
The commercial industry utilises many different types of systems in order to sell products and keep selling products, with one of the most prominent being rewards systems. 
Rewards systems in the commercial industry vary from paper based frequent visit cards to digital points tracking systems. 
The following sections discuss some of these systems, in particular the digital renditions and how they are achieving user engagement, as well the benefits to the business.

\subsubsection{Paper Based}
Paper based rewards cards are on their way out with most companies turning to digitally tracking a customers purchases. 
Regardless it is still worth mentioning these systems as they form the basis for their digital cousins.

\par
The basics of these paper based rewards are as follows; customers are given a card that gets stamped or punched every time the customer completes a task (commonly making a purchase), once the customer meets a certain requirement they are rewarded, the reward depends on the business but generally consists of a discount on the next purchase. 
This form of reward is referred to as an extrinsic reward, section~\ref{sec:extrinsic} will discuss this further.

\subsubsection{Digital}
Bringing the above mentioned rewards systems into the digital world allows the implementers to incorporate some extra techniques; one such technique being the introduction of gamification (see section~\ref{sec:gamification}). 
An example of this is Starbucks rewards system where 'achievements' are earned, such as free refills, after 'leveling up' your reward card through making purchases~\citep{gamifying-intelligent-environments} and interacting with the application that is available alongside the card.

\begin{figure}[!ht]
	\centering
	\begin{minipage}{.5\textwidth}
	  \centering
	  \includegraphics[width=.6\linewidth]{./images/clubCard-boost}
	  \captionof{figure}{Boost's Club card~\citep{boost}}
	  \label{clubCard-boost}
	\end{minipage}%
	\begin{minipage}{.5\textwidth}
	  \centering
	  \includegraphics[width=.6\linewidth]{./images/clubCard-sca}
	  \captionof{figure}{Super Cheap Auto's Club+ card~\citep{sca}}
	  \label{clubCard-sca}
	\end{minipage}
\end{figure}

\par
These types of cards are used throughout many different businesses in order to track what users purchase, which ultimately benefits the business. 
Because using these cards is optional it can be seen that these systems must offer some way of motivating a user to ensure ongoing interaction.

\subsection{Community}
Lately there have been many different applications being published, each with the aim to modify the behavior of the general population. 
The following sections analyse some of these applications and discuss the methods used to increase a users interaction with the application.

\subsubsection{I-GEAR}
I-GEAR is a project currently being run by the University of Luxembourg in order to encourage commuters to modify their transport behaviors to ultimately reduce congestion~\citep{igear,igear-2}. 
Researchers on the I-GEAR project are looking into how gamification techniques can be used to make the project successful.

\subsubsection{RunKeeper}
RunKeeper is a fitness application dedicated to motivate people to get fit. 
RunKeeper measures and tracks its users physical activities, helping them set goals and fulfill achievements. 
RunKeeper pits its users against each other in order to utilise their natural desire to win~\citep{desire-to-win}.

\begin{figure}[!ht]
\centering
\begin{minipage}{.5\textwidth}
  \centering
  \includegraphics[width=.5\linewidth]{./images/application-runkeeper-goals}
  %\captionof{figure}{RunKeeper achievements~\citep{runkeeper}}
  \label{application-runkeeper-achievements}
\end{minipage}%
\begin{minipage}{.5\textwidth}
  \centering
  \includegraphics[width=.5\linewidth]{./images/application-runkeeper-socialnetworking}
  %\captionof{figure}{RunKeeper networking~\citep{runkeeper}}
  \label{application-runkeeper-social}
\end{minipage}
\caption{RunKeeper - achievements (left) and networking (right)}
\end{figure}


\subsubsection{QuitNow!}
QuitNow! is a smartphone application designed to aid its users through the process of quitting smoking.
In order to motivate its users QuitNow! tracks their success in the form of achievements, rewarding them with badges in order to ``motivate you[its users] to achieve your goal''\citep{quitnow}. 
QuitNow! also implements an internal social networking aspect in order to utilise our natural competitive desire~\citep{bread-and-games}.\\

\begin{figure}[!ht]
\centering
\begin{minipage}{.5\textwidth}
  \centering
  \includegraphics[width=.6\linewidth]{./images/application-quitnow-achievements}
  \captionof{figure}{QuitNow! achievements~\citep{quitnow}}
  \label{application-quitnow-achievements}
\end{minipage}%
\begin{minipage}{.5\textwidth}
  \centering
  \includegraphics[width=.6\linewidth]{./images/application-quitnow-social}
  \captionof{figure}{QuitNow! networking~\citep{quitnow}}
  \label{application-quitnow-social}
\end{minipage}
\end{figure}

QuitNow!'s success can be measured through the feedback received through their Google Play store page, such as: \\

\indent\textbf{``Love it! I like that it resets the time if you slip and have a cigarette. It frustrated me to have to start over again, which I think helped me quit faster. Love the social part and the achievements you can earn. It's just great! Almost six months so far!''}~\citep{quitnow},\\

\indent\textbf{``I love seeing how many cigarettes I've not smoked and the money I've saved. When I need an extra push the image galley is a nice reminder of why I'm quitting.''}~\citep{quitnow}\\

Despite these successes however some concerns have been raised in relation to the social networking aspect, evident by a users feedback of:\ 

\indent\textbf{``Too many people being racist or taking mick out of disabled people or just purely trolling quite bad when all peopld want Is support when quitting!!!''}~\citep{quitnow}
The potential pitfalls of social media are explored further in section~\ref{sec:social-media}.

\subsubsection{Vivid}
Vivid is a light show that is occasionally played in Sydney, with its last appearance running from the 24th of May until the 10th of June 2013.
Vivid uses large scale attractions, such as colouring the entire city, music concerts, and debates from well known public figures~\citep{vivid}. 
There is not a lot of opportunities for the public to interact with the systems in place, for this reason Vivid solely relies on the scale of attractions to motivate and interest people.

%\newpage
\section{City Evolutions}
Section~\ref{sec:city-evolutions-project} introduced the City Evolutions project and its aims, this section will critically analyse the implementation of the project.
This section will additionally make some suggestions at new approaches for the project aimed at increasing the amount of interaction between the users and the system.

\subsection{Analysis}\label{sec:analysis}
The hardware installed on Watt Street is sufficient at this point in time to support high amounts of interaction, instead this section will focus on the current applications that are running in order to point out the good and bad points. 

\par
Every site along Watt Street has a QR code associated with it (Figure~\ref{QR-threkeld-barcode}), when scanned these QR codes direct the user to information about the site. 
\begin{figure}[ht!]
	\centering
	\includegraphics[width=40mm]{./images/qrcode-threlkeld}
	\caption{QR Code - Threlkeld Speak}
	\label{QR-threkeld-barcode}
\end{figure}
The level of interaction is primarily measured through the amount of scans that have been logged per QR code. 
This method was chosen because of the relationship between how interested a user is with the content at a site and their desire to learn more about it.

\subsubsection{Results}
Figure~\ref{QR-overall-access-overTime} shows the overall activity since the projects launch in June. 
It can be seen that initial interest was high, the peaks can be explained by the amount of advertisement the project saw in its early stages. 
Unfortunately a drop in the number of scans can be seen representing little to no activity at any of the sites from August onwards.
Without the advertisement it appears as if the content delivered by the project is not enough to keep users interested or motivated.

\begin{figure}[ht!]
	\centering
	\includegraphics[width=150mm]{./images/OverallQRCodeDownload}
	\caption{QR Code - Overall scans}
	\label{QR-overall-access-overTime}
\end{figure}

\begin{table}[ht!]
	\centering
	\begin{tabular}{|c|c|c|}\hline
		\textbf{Application} & 	\textbf{Number of Scans} & \textbf{Interactivity (see table~\ref{table:interactivity-description})} \\\hline
		Macquaries Chest & 2 & Low\\
		Car Culture & 3 & None\\
		Lansescape & 34 & Medium\\
		Coal Collector & 36 & Medium\\
		Dreaming & 3 & None\\
		Major Morisset & 36 & Low\\
		Evolutions & 6 & None\\
		Shanghai & 2 & Medium\\
		Boom \& Bust & 2 & Low\\
		Threlkeld Speak	&	132 & Medium\\\hline
	\end{tabular}
	\caption{City Evolutions Applications and their popularity}
	\label{table:application-popularity}
\end{table}

\par
It is beneficial to analyse each application in order to get a better of representation of what types of applications motivated users the most. Table~\ref{table:application-popularity} shows the applications including a breakdown of the types of interaction utilised in each application (defined in table~\ref{table:interactivity-description}).

\begin{table}[ht!]
	\centering
	\bgroup
	\def\arraystretch{2.5}%  1 is the default, change whatever you need
	\begin{tabular}{|l|l|}\hline
		\textbf{Level} 	& \textbf{Description}\\\hline
		None 	& \pbox{20cm}{No user input is taken at all, generally a slideshow or movie} \\\hline
		Low 	& \pbox{20cm}{Generally a single has basic control over the application\\ 
				  			  i.e. User makes movement to trigger the application to take action.} \\\hline
		Medium 	& \pbox{20cm}{Users have basic control over the application and are given constant\\
							  feedback, the more users the more feedback is received.}\\\hline
		High 	& \pbox{20cm}{Users have fine grained control over the application,\\ 
							  with constant feedback and a reason to come back to the application\\
							  i.e. scores are saved, competitions are run, etc.}\\\hline
	\end{tabular}
	\egroup
	\caption{Description of interactivity levels used in table~\ref{table:application-popularity}}
	\label{table:interactivity-description}
\end{table}

\par
\textbf{Threlkeld Speak}\\
Table~\ref{table:application-popularity} shows that the Threlkeld Speak application was the most popular application on Watt Street.
Threlkeld Speak consisted of a set of movies that represented words from the Awakabal language. The aim was to educate users on the history of this native language though visual displays of the words.
Users were given control over which movie-word pair was displayed from the projector through the use of an Android application.
\par
\begin{figure}[ht!]
	\centering
	\includegraphics[width=150mm]{./images/threkeld-scans}
	\caption{QR Code - Threlkeld Speak scans}
	\label{QR-threkeld}
\end{figure}
This application represents an application with a medium level of interaction and had no sense of competition, which could be one of the reasons for its decline in popularity (figure~\ref{QR-threkeld}).

It is worth noting that the application ceased running on the first week of October 2013 due to building works rendering it impossible to project onto the church; however this doesn't account for the earlier decline.

As can be seen in figure~\ref{QR-OS-access} it was a mistake to not make the control application available on iOS as the two operating systems have an almost equal share in users.


\begin{figure}[ht!]
	\centering
	\includegraphics[width=50mm]{./images/OSAccess}
	\caption{QR Code - Operating system spread}
	\label{QR-OS-access}
\end{figure}

\par
\textbf{Trigger Applications}\\
Macquaries Chest, Major Morisset, and Boom \& Bust were all applications that featured an animation being triggered relative to the amount of motion in the area. 
This motion was picked up through the use of motion detecting cameras. 
The applications generally featured a calm movie while there was no motion in the area, a subtle movie would be played if someone walked by the camera, and if people stayed in the area then a more significant movie was played.

These applications showcased a subtle rewarding system similar to that found in games, where users are rewarded with movie clips that tell a story. 
Unfortunately it seems that these applications were not as popular as they were expected, with the exception of Major Morisset. 
It can be inferred that the reason for the difference in success between Major Morriset and the other applications of its type could be due to the actual content that is shown.
More research needs to be conducted to back this up but if it is found to be true then an update in content would be enough to revitalise these applications.

\par
\textbf{Lanescape}\\
Lanespace is an application that makes use of two projectors, two cameras, and an audio system.
Music is played and a dance floor is projected onto the ground, encouraging users to dance.
When users interact with this application their silhouettes are projected onto the wall for them to watch.
The more people to join in the dancing, the more intense the visual projection becomes. 
This application is the only application that makes use of sound, which together with the highly visual component makes a very appealing application. 
Unfortunately there has been a decline in the use of this application which can again be put down to the lack of changing content. 

\par\textbf{Shanghai}\\
Shanghai is an application that makes use of motion detection cameras to allow users to move a boat left and right.
This allowed for a rendition of the classic Pong game where users had to use the boat to bounce a ball around the projection in order to gain points.
When enough points where collected the application would show a video relevant to the history of Newcastle, where a sailor was conscripted onto the boat.

Due to the camera not being programmed to recognise each interacting individual, users were forced to work together as a single player to achieve their common goal.

The Shanghai application showcases a few of the techniques mentioned in section~\ref{sec:techniques-to-enhance}, such as:
\begin{itemize}
	\item{Points;}
	\item{Rewards;}
	\item{Achievements; and}
	\item{Ease of use}
\end{itemize}
However the application was never put into production due to it not performing correctly in the environment, namely it was not cross-platform and thus could not be run on the computer that it was designated. Hence the low indication of user interaction shown in table~\ref{table:application-popularity}.
This highlights the importance of creating applications that are capable of running across different platforms.

\par\textbf{Coal Collector}\\
Coal collector is a game played by two teams of users on their smartphones, each team consists of one to four players. 
Players control a ship and move coal around while avoiding obstacles.
Table~\ref{table:application-popularity} shows that the application was relatively successful and it can be assumed that this was due to the competitive nature of the application.
Future additions to this application would be the addition of a persistent leader board to keep track of each individuals contributions. 
The games content itself will need constant refreshing due to the repetitive nature of the game.

\par
\textbf{No Interaction}\\
Evolutions, Dreaming, and Car Culture are identified in table~\ref{table:game-rewards} as having no interaction, this is because they were simply projections of videos and slide shows. 
These segments were set up to showcase Newcastle history but the users had no way of giving their input. 
It can be seen that users did not feel the need to scan the related QR codes and it is assumed to be due to the lack of interactivity present.

\subsubsection{Summary}
At present each site in the City Evolutions Project is considered a separate site, meaning there is no relationship between each site aside from the type of content they are displaying.
None of the applications utilise a users competitive nature, none of the applications have a concept of a persistent score or a player, treating each person who engages with the system as a new person.
Interactivity between a user and the project is generally limited to simple trigger actions, if at all.

\subsection{Proposal}
The suggestions mentioned in section~\ref{sec:analysis} may not be enough to revitalise the project, therefor this section will outline a larger scale change to aid in enhancing the user interaction with the City Evolutions Project.

I propose that we discard the idea of separate sites, each site should complement the next. 
This would need completely new applications at all sites, since new applications need to be developed, I propose that applications shouldn't always be available at the same time as they are at the moment.
Applications should change times and even sites, this is to stop people getting tired of the same content.

\subsubsection{Smartphone Application}
To get proposal this to come together nicely I have designed a cross-device mobile application that can interact with all the sites on Watt Street.
Figures~\ref{iphone-interface-startscreen},~\ref{iphone-interface-menu},~\ref{iphone-interface-leaderboard}, and~\ref{iphone-interface-map} show some example functionality of this application.
Figure~\ref{iphone-interface-startscreen} displays the first interface that the user is presented with. `Whats on in Newcastle' leads the user to figure~\ref{iphone-interface-menu}, here the user can see what attractions are currently running. 
The user can select an attraction and the application will guide them to the correct site, through use of the phones global positioning system (GPS).
Once the user in the sites vicinity the application will enable them to join the session that is currently, this will be done by either presenting them with an on screen controller or by indicating the alternate interaction device (camera, microphone, etc). 
This means that any interaction between a smartphone and an application at a site can be done through this application, alleviating the need to create new applications constantly while also alleviating the need for the user to swap applications.

\begin{figure}
\centering
\begin{minipage}{.5\textwidth}
  \centering
  \includegraphics[width=.6\linewidth]{./images/iphone-interface}
  \captionof{figure}{Application start screen}
  \label{iphone-interface-startscreen}
\end{minipage}%
\begin{minipage}{.5\textwidth}
  \centering
  \includegraphics[width=.6\linewidth]{./images/iphone-interface-menu}
  \captionof{figure}{Application Menu}
  \label{iphone-interface-menu}
\end{minipage}
\end{figure}

\subsubsection{Global scoring}
Further I propose that every application keep track of users, introducing customisable avatars and score boards. Keeping with the theme of connecting all the applications in some form, I propose a global leader board be introduced. 
In this new system users can gain points at each site, these points are local to each site and also contribute to the players global score.
Once a defined amount of points are achieved at a site the rate at which points are earned will slow down, encouraging users to move to the next site or come back another day to try a new application.
Seen in figure~\ref{iphone-interface-leaderboard} is the aforementioned leader board with figure~\ref{iphone-interface-map} showing a site (red) where a player has exhausted their ability to gain points quickly, along with a nearby site (green) that has not yet been attempted.

\begin{figure}
\centering
\begin{minipage}{.5\textwidth}
  \centering
  \includegraphics[width=.6\linewidth]{./images/iphone-interface-leaderboard}
  \captionof{figure}{Application leader board}
  \label{iphone-interface-leaderboard}
\end{minipage}%
\begin{minipage}{.5\textwidth}
  \centering
  \includegraphics[width=.6\linewidth]{./images/iphone-interface-map}
  \captionof{figure}{Application locations map}
  \label{iphone-interface-map}
\end{minipage}
\end{figure}


\subsubsection{Social Networking}
This new application will have social networking aspects, integrating with popular social networking sites such as Twitter and Facebook.
This integration can provide the advertisement needed to keep the project alive.
As seen in figure~\ref{QR-overall-access-overTime}, without advertisement the project will has slowly become less popular.

\par
Along with the advertising potential from social media, the previously mentioned leader board (figure~\ref{iphone-interface-leaderboard}) can now have the ability to be filtered.
This allows users to compare their scores directly with the people that matter to them, as well as the best of the best that Watt Street has to offer.

\subsubsection{Potential Applications}
The following section outlines some applications that can be set up in the new proposed environment.

\par\textbf{Multi player Pong}\\
A new application that can be set up at one of the sites is a multi player Pong style game (figure~\ref{application-multiPlyerPong}). 
The application has been shown with generic looking objects so it is not tied to one set of content, i.e. the blue paddles can be changed to any type of content deemed fit for the story trying to be conveyed.
This application implements many of the points discussed in section~\ref{sec:techniques-to-enhance}. 
The application is competitive, allowing more than one player and up to eight players to compete with each other.

\begin{figure}[ht!]
	\centering
	\includegraphics[width=125mm]{./images/MultiPlayerChurchGame}
	\caption{Multiplayer pong-style game to be played at the church.}
	\label{application-multiPlyerPong}
\end{figure}

To the far right we can see the sites leader board, the players shown here (shown by chosen name and avatar) are relative to the current game being played. 
Objects can come out of architectural aspects of the building that the application is projected against, such as the church door or windows.
These objects can be collected for points, or to modify the applications state by changing the size of the paddles or the ball.
Each time a player misses the ball and it goes out of bounds, that players paddle is made smaller until its non-existent and the player removed from the game. This function is to ensure new players can start playing if there is a queue, if there is no queue then the removed player may join back in.

\par\textbf{Multi player Pacman}\\
With the use of smartphones there is no need to limit the project to projections that can only be shown at night time.
The phone itself can act as the output device for an application that is set up on Watt Street. 
This leaves it open for games such as the one conceptualised in figure~\ref{application-pacman}.
This proposed application utilizes the phones GPS sensors to get the users to play a large scale hide and seek type of game.
Users can choose to be displayed on the map as either a Pacman character or they can use their own customised avatar.
Users must find other users who are participating in the game and can even find other items that can be placed at popular locations, thus also incorporating a scavenger hunt aspect.
\begin{figure}[ht!]
	\centering
	\includegraphics[width=125mm]{./images/pacman}
	\caption{Pacman style game to be played throughout Watt Street}
	\label{application-pacman}
\end{figure}


%\newpage
%\section{Discussion}
%
%Give an extended and detailed discussion of your study. explain what we can learn from your study.
%

\section{Conclusion}
To summarise, systems need user engagement and interaction, there is no point creating a system that no one is going to use.
Achieving sufficient levels of user engagement requires a thorough definition of the systems aims followed by methods to achieve those aims. 
An example from this paper is the City Evolutions Project, its aims were to educate tourists and citizens about the history and culture of Newcastle.
After this definition was made and a project was designed, the approach I propose is to add game mechanics to the project in order to motivate a user to interact with the system.

For a user to be motivated to engage with a system, there must be interaction. The slide shows and videos played on Watt Street attracted the least amount of engagement when compared to applications that a user could pick up and use.

A successfully engaging application will leave a user intrinsically rewarded and allow both the product owner and its users to obtain the most value out of the the system.

\subsection*{Acknowledgements}
I am grateful to Dr. Yuqing lin and A/Prof Stephan Chalup for their guidance through this project.
%
\vskip 0.2in
\newpage
\bibliographystyle{apalike}
\bibliography{./literature.bib}


\end{document}
